\documentclass[12pt]{book}

\usepackage{listings}

\title{Introduction to Tools for Computer Science and Software Development}
\author{Ian Ooi, Amelia Peterson}

\begin{document}
	\begin{titlepage}
		\maketitle
	\end{titlepage}
	\tableofcontents
	
	\chapter{Setting up your System}
		Picking a platform is largely a personal choice, but some factors to consider include:
		\begin{itemize}
			\item Target platform (What is this going to be run on?  What is your TA/Professor using?  If for production, is the user on Windows, Mac, Linux...?)
			\item How comfortable are you with each?  (Have you ever used Linux?  Have you used Windows?  Have you developed software on the platform?)
			\item What tools do you know about for the platform? (We're hoping to help a bit here, but do you have any preferred tools you use already?  Do you have help available from a friend, TA or Professor with a specific software?  Is the tool going to make life a lot easier?)
			\item What is going to have the best tools available? (If you only need to make Windows GUIs, developing on Windows is probably a good idea.)
		\end{itemize}
		%Your development environment is a personal choice.  Some factors to consider include target platform, your familiarity with different software and environments, and the tools available to you in each environment.  We're proponents of Linux development, but there are situations where a different system may be ideal.
		\section{Linux}
			Benefits of using Linux:
			\begin{itemize}
				\item Software and tools generally available for free
				\item Flexible, configurable systems
				\item Numerous tools and programs available
				\item Numerous tools geared toward software development
				\item Community help and support
				\item Extensive, \emph{built-in} documentation
				\item You can fix your own problems (Windows/Mac you generally need to just reinstall or, for Macs, take it to a professional)
				\item Unix itself can be your IDE
			\end{itemize}
			\subsection{Choosing a Distro}
				Popular distributions include:
				\begin{itemize}
					% more or less?
					\item Debian
					\item Ubuntu (based off Debian)
					\item Linux Mint (based off Debian)
					\item Arch Linux
					\item Fedora
					\item OpenSUSE
					\item FreeBSD
					\item Slackware
					\item Gentoo
				\end{itemize}
				Picking is, again, a matter of preference, but there are tradeoffs.  {\bf If you are new to Linux, it is probably best to use Linux Mint, Ubuntu or Debian.}  All three come prepackaged with most everything you need to have your system set up with a GUI, internet access and media (sound, CDs, DVDs etc.) access.  When you are more experienced however, remember that these can come with features you distinctly don't want, or want to replace (such as window managers).
				
				Arch Linux has some of the best documentation around, but is distinctly less beginner friendly (you will not start off with simple internet access, a GUI, media, and a number of other amenities).  Gentoo and Slackware are on the far end of the spectrum and are really only for people who know what they're doing or are distinctly masochistic.
				%There are numerous different distributions, and it really (again) comes down to preference and what you want from your system.  Debian and Debian based distributions (e.g. Ubuntu and Linux Mint) are very beginner friendly and come pre-packaged with most of the tools you'll need for some of those basic operations you take for granted (displaying GUIs, adjusting volume and brightness, and connecting to the internet for instance).  They are somewhat less flexible as a result, or require more work to configure if you want a different significantly different setup than one of the pre-packaged ones.  Debian based distributions are excellent, however, and generally you don't need much beyond the pre-packaged configuration.
				
				%If you're feeling adventurous or just want to try something different, there are numerous other distributions, such as Arch Linux, which is a bit less beginner-friendly and more difficult to configure.  
			\subsection{Text Editors}
				
		\section{Windows}
			Setting up Windows should be a simple matter of either:\begin{itemize}
				\item[A)] Buying a computer with Windows on it, or...
				\item[B)] Buying a license for Windows, popping the disc in (or whatever other installation media you are using) and following the directions given
			\end{itemize}			
			
			%TODO add section reference to IDEs
			From here, you can choose what you want to develop with.  If you want to use Microsoft Visual Studio, simply purchase a license (or get one through school, work etc.) and install.  Other IDEs (discussed later) are also available for C/C++ development as well as most other languages.  Some C/C++ IDEs include: \begin{itemize}
				\item Eclipse
				\item %TODO ide list
			\end{itemize}
			
			If you don't want a full IDE, there are also numerous editors, compilers and debugging tools available as individual programs.
			\subsection{Editors}
				Choosing an editor is truly a matter of personal preference, depending on the features you're looking for.  Some, like Vim or Emacs are very feature rich and extensible, with the tradeoff being a high learning curve.  Others, such as using Notepad, are very simple, but can be difficult because they lack useful features like syntax highlighting.  In general, you want something with (at the very least) syntax highlighting and often an option for soft tabs (replacing tabs with a number of spaces, usually 4).  C++ doesn't need soft tabs per se, but it's useful for preserving your code formatting across systems and programs.  Soft tabs becomes more necessary in situations where whitespace (spaces, tabs, newlines) become meaningful, such as in Python or Haskell.  Some will also include extra features, such as built in debugging, compiling or running programs, and editing extras such as macros.
				\begin{itemize}
					\item Notepad++
					\item SciTE
					\item Vim
					\item Emacs
					\item Code::Blocks
					\item Notepad2
				\end{itemize}
			\subsection{Compilers and Interpreters}
				You will need a suitable compiler or interpreter for whatever language you are developing in.  Notable compilers for C++ are GCC (g++ is the C++ compiler) and MSVCC, which comes with Microsoft Visual Studio.
				
				GCC can be used by installing MinGW, Cygwin, or another tool which either mimics or 
		\section{Mac}
			Coding on a Mac can be much the same as coding on a Linux system.  Both are (since Mac OS X) Unix based, and as such, most of the tools available.  In general, you can set up your system the same way a Linux user would, though some additional tools and software may be available.  %TODO get help from someone who uses a mac...
			
	\chapter{Basic Terminal Usage}
		Alright then, wow that was lots of words.  Instead, have lists of commands you'll need.
		\section{Linux/Mac}
			When using the terminal, help text and man pages are available for most tools, commands and software, as well as extensive information on the internet.
			
			Manual pages can be accessed by running \verb+man <command>+, so, for example, to find help on the command \verb+grep+, you would run \verb+man grep+.  This also works for built in C++ and C functions, e.g. \verb+man printf+.
			
			Command line tools usually include help text as well, accessed either through the \verb+--help+ flag (e.g. \verb+ls --help+) or from the \verb+-h+ flag (e.g. \verb+sudo -h+).
			\subsection{Moving Around/Browsing your files}
				\begin{itemize}
					\item \verb+ls+: List the files in a directory.  Common flags are \verb+-a+ to list hidden files and \verb+-l+ to print extra information.
					
						\verb+ls -a+
						
						\verb+ls -al+
					\item \verb+cd+: Move to a directory.  Note \verb+..+ is the directory above and \verb+.+ is the current directory.
					\item \verb+mv+ Move file or folder to a different directory. (Will automatically recurse for folders!)
					\item \verb+cp+ Copy a file or folder to a different directory. (Will not automatically recurse, must specify with \verb+cp -r+.)
				\end{itemize}
			\subsection{Permissions}
				\begin{itemize}
					\item \verb+sudo+
					\item \verb+su+
					\item \verb+chmod+
					\item \verb+chown+
					\item \verb+gksudo+ GTK frontend for sudo, creates a window which prompts for the password instead of through command line.  Useful for executing from the run dialog (Alt+F2) or things like dmenu.
				\end{itemize}
			\subsection{Other useful commands}
				\begin{itemize}
					\item \verb+grep+ Regular expressions
					\item \verb+cat+
					\item \verb+less+
					\item \verb+more+
					\item \verb+alias+
					\item \verb+man+
					\item \verb+apt-get+
					\item \verb+aptitude+
					\item \verb+dpkg+
					\item \verb+screen+ Terminal multiplexing.  Lets you have multiple terminals in one, as well as split screens, move between them and detach to save sessions for later.  \emph{Screen will also allow you to connect to a serial device.}
					\item \verb+tmux+ A different terminal multiplexor with some additional features for multiplexing, but will not connect to serial devices.
					\item \verb+pacman+
				\end{itemize}
			
			\subsection{Configuring}
				If you use bash as your shell (if you are on Debian/Ubuntu/Linux Mint bash is the default), you can configure it using your \verb+.bashrc+ file, found in your home directory (\verb+/home/yourusername/.bashrc+, equivalent to \verb+~/.bashrc+).
				
				Any valid bash shell commands can be written and they will execute.  For example, \verb+alias cp='cp -v'+ will cause cp to always run with the verbose (\verb+-v+) flag.
					
		\section{Windows}
			\subsection{Moving Around/Browsing Files}
				\begin{itemize}
					\item 
				\end{itemize}
	\chapter{Text Editors and IDEs}
		\section{Command Line Editors}
		\section{GUI editors/IDEs}
		
	\chapter{Compiler Usage}
		\section{Command Line Compilers}
		\section{GUI based compilers}
		\section{Linking Libraries}
		\section{Cross-Compiling}
		
	\chapter{Debugging}
		\section{Types of Errors}
			\subsection{Compile-time}
				\begin{itemize}
					\item 
				\end{itemize}
			\subsection{Runtime}
				\begin{itemize}
					\item Segmentation Fault (Seg Fault): Your program tried to access memory that it couldn't, or otherwise did something with memory it shouldn't have.  This is usually caused by something wrong with pointers, your other logic, or a stack overflow, where your program ran infinitely or used too much memory (infinite/large sized arrays/vectors etc.).
					\item 
				\end{itemize}
		\section{Debugging Tools}
			\subsection{GDB}
			\subsection{Visual Studio Debugger}
		\section{Debugging Memory}
			\subsection{Valgrind}
			\subsection{Dr. Memory}
		\section{Debugging Performance}
		\section{Understanding Compiler Errors}
			
	\chapter{\LaTeX}
		There comes a time in most science, math and engineering students' lives when they need to write pages of equations. \LaTeX is made for typesetting documents, including equations, tables, pictures and text, and is perfectly suited for pages of proofs, lab reports, and general documents.  It may look intimidating at first, because you're writing markup to create your document, and it seems a bit like code, but it makes formatting much simpler and creates neater, more professional documents.
		
		The basic structure of a \LaTeX document looks like this:
		\lstinputlisting[language=TeX]{examples/latex/latexbasics.tex}
	\chapter{Build Systems}
		\section{Makefiles}
		\section{CMake}
		
	\chapter{Source Control}
\end{document}
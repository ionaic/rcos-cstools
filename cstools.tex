\documentclass[12pt]{book}

\title{Introduction to Tools for Computer Science and Software Development}
\author{Ian Ooi, Amelia Peterson}

\begin{document}
	\begin{titlepage}
		\maketitle
	\end{titlepage}
	\tableofcontents
	
	\section{Setting up your System}
		Your development environment is a personal choice.  Some factors to consider include target platform, your familiarity with different software and environments, and the tools available to you in each environment.  We're proponents of Linux development, but there are situations where a different system may be ideal.
		\subsection{Linux}
			\subsubsection{Choosing a Distro}
				There are numerous different distributions, and it really (again) comes down to preference and what you want from your system.  Debian and Debian based distributions (e.g. Ubuntu and Linux Mint) are very beginner friendly and come pre-packaged with most of the tools you'll need for some of those basic operations you take for granted (displaying GUIs, adjusting volume and brightness, and connecting to the internet for instance).  They are somewhat less flexible as a result, or require more work to configure if you want a different significantly different setup than one of the pre-packaged ones.  Debian based distributions are excellent, however, and generally you don't need much beyond the pre-packaged configuration.
				
				If you're feeling adventurous or just want to try something different, there are numerous other distributions, such as Arch Linux, which is a bit less beginner-friendly and more difficult to configure.  
		\subsection{Windows}
			
		\subsection{Mac}
			Coding on a Mac can be much the same as coding on a Linux system.  Both are (since Mac OS X) Unix based, and as such, most of the tools available 
			
	\section{Basic Terminal Usage}
		\subsection{Linux/Mac}
		\subsection{Windows}
	
	\section{Text Editors and IDEs}
		\subsection{Command Line Editors}
		\subsection{GUI editors/IDEs}
		
	\section{Compiler Usage}
		\subsection{Command Line Compilers}
		\subsection{GUI based compilers}
		\subsection{Linking Libraries}
		\subsection{Cross-Compiling}
		
	\section{Debugging}
		\subsection{Types of Errors}
		\subsection{Debugging Tools}
		\subsection{Debugging Memory}
		\subsection{Debugging Performance}
		\subsection{Understanding Compiler Errors}
		
	\section{\LaTeX}
		There comes a time in most science, math and engineering students' lives when they need to write pages of equations.	
	
	\section{Build Systems}
		\subsection{Makefiles}
		\subsection{CMake}
		
	\section{Source Control}
\end{document}
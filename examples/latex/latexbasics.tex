% this is a comment, it won't be in your document
% include documentclass at the top of every LaTeX document
% 12pt is an argument telling it to make the general font 12pt
% article is just a basic type of document for most purposes
\documentclass[12pt]{article}

% specify a title
\title{My TAs Love Me Because They Can Read My Work}

% specify an author
\author{Chuck Finley}

% specify the date, \today just inserts the current date
\date{\today}
% \date{October 5, 2012} %specifies a specific date instead

% start the body of the document
\begin{document}
    % tell LaTeX to actually make a title, it will
    % change the size and center it for you
    \maketitle
    % these are still comments
    You can write whatever you want here, just like
    you would in Word, LibreOffice or OpenOffice.  
    One major difference is that LaTeX removes 
    extraneous (more than 1) spaces    such   as   
    these.
	
    A single empty line will start a new paragraph 
    and indent it for you. The first paragraph is not 
    indented by default, don't worry about it. It bothered 
    me at first too, but it's really going to be ok.
	
    LaTeX (\LaTeX) commands start with a backslash (\). 
    \emph{This text is emphasized, similar to italics.} 
    \textbf{This text is bold.}  You have to escape 
    backslashes, such as \\.  The same works to type \%.
\end{document}